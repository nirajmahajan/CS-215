
\documentclass[11pt]{article}
\usepackage {fullpage,amsmath,amssymb,lipsum,url}

\usepackage[scaled]{helvet}
\renewcommand*\familydefault{\sfdefault} %% Only if the base font of the document is to be sans serif
\usepackage[T1]{fontenc}

\title {CS 215 : Data Analysis and Interpretation : Assignment}

\author {Instructor : Suyash P. Awate}

\date {Due Date : 8 Nov 2019, Fri, 11:55 pm; Maximum Points 40}

\begin {document}

\maketitle

\setlength{\parindent}{0cm}

{\bf Submission Instructions: } \\

$\bullet$ IITB and CSE have zero tolerance to plagiarism. \\

$\bullet$ If you submit the solution to the assignment, you agree that every line of code and every line in the report is your (or your group's) own,
and hasn't been copied from any other source offline or online. \\

$\bullet$ If you submit the solution to the assignment as a group and any member of the group is found to have committed plagiarism, then the full
penalty will be applicable to every member of the group. \\

$\bullet$ For the sake of effective learning, if you submit the solution to the assignment as a group, then each member of the group agrees to have
participated fully (100\%) in performing every part of every question in the assignment. \\

$\bullet$ Submit your solution, i.e., (i)~the code, (ii)~the results, e.g., graphs or other data, and (iii)~the report (in Adobe PDF format), for each
question, through moodle. Put the code within the folder ``code'', all results in the folder ``results'', and the report in the folder ``report''. You
may make subfolders for each question in the assignment. \\

$\bullet$ Submit a single zip file that contains the solution to each problem below in a separate folder. \\

$\bullet$ To get any possible partial credit for the code, ensure that the code is very well documented. \\

$\bullet$ To get partial credit for the derivations, include all derivation steps in their full details. \\

$\bullet$ To avoid non-deterministic results in each program run, and to make the results reproducible during test time, use rng(seed) where seed is a
fixed hard-coded integer in your code. \\

$\bullet$ While submitting the results of this question, use the "publish" feature in MATLAB. \\

$\bullet$ {\bf 5 points} for submission in the proper format. \\

$\bullet$ If you feel there is a typo in the question, please make suitable assumptions, consistent with those in the question, and proceed to solve
the problem. Also, please let the TAs or the instructor know. \\


\begin {enumerate}
  
\item % Bayesian estimation
  
  (10 points) Use the Matlab function randn() to generate a data sample of $N$ points drawn from a Gaussian distribution with mean $\mu_{\text{true}}
  = 10$ and standard deviation $\sigma_{\text{true}} = 4$.  Consider the problem of using the data to get an estimate $\widehat \mu$ of this Gaussian
  mean, assuming it is unknown, when the standard deviation $\sigma_{\text{true}}$ is known.
  
  Consider using one of the two prior prior distributions on the mean: (i)~a Gaussian prior with mean $\mu_{\text{prior}} = 10.5$ and standard
  deviation $\sigma_{\text{prior}} = 1$ and (ii)~a uniform prior over $(9.5, 11.5)$.
  
  Consider various sample sizes $N = $ 5, 10, 20, 40, 60, 80, 100, 500, $10^3, 10^4$.  For each sample size $N$, repeat the following experiment $M
  \ge 100$ times: generate the data, get the maximum likelihood estimate $\widehat \mu^{\text{ML}}$, get the maximum-a-posteriori estimates $\widehat
  \mu^{\text{MAP1}}$ and $\widehat \mu^{\text{MAP2}}$, and measure the relative errors $| \widehat \mu - \mu_{\text{true}} | / \mu_{\text{true}}$ for
  all three estimates.
  
  $\bullet$ Plot a single graph that shows the relative errors for each value of $N$ as a box plot (use the Matlab boxplot() function), for each of
  the three estimates.
  
  $\bullet$ Interpret what you see in the graph. (i)~What happens to the error as $N$ increases~? (ii)~Which of the three estimates will you prefer
  and why~?
  
\item % Transformation and Bayesian estimation
  
  (10 points) Use the Matlab function rand() to generate a data sample of $N$ points from the uniform distribution on $(0,1)$. Transform the
  resulting data $x$ to generate a transformed data sample where each datum $y := (-1 / \lambda) \log (x)$ with $\lambda = 5$. The
  transformed data $y$ will have some distribution with parameter $\lambda$; what is its analytical form~? Use a Gamma prior on the
  parameter $\lambda$, where the Gamma distribution has parameters $\alpha = 5.5$ and $\beta = 1$.
  
  Consider various sample sizes $N = $ 5, 10, 20, 40, 60, 80, 100, 500, $10^3, 10^4$.  For each sample size $N$, repeat the following
  experiment $M \ge 100$ times: generate the data, get the maximum likelihood estimate $\widehat \lambda^{\text{ML}}$, get the Bayesian
  estimate as the posterior mean $\widehat \lambda^{\text{PosteriorMean}}$, and measure the relative errors
  $| \widehat \lambda - \lambda_{\text{true}} | / \lambda_{\text{true}}$ for both the estimates.
  
  $\bullet$ Derive a formula for the posterior mean.
  
  $\bullet$ Plot a single graph that shows the relative errors for each value of $N$ as a box plot (use the Matlab boxplot() function), for both the
  estimates.
  
  $\bullet$ Interpret what you see in the graph. (i)~What happens to the error as $N$ increases~? (ii)~Which of the two estimates will you prefer and
  why~?
  
\item % Gaussian fit to circular data
  
  (5 points) Consider a 2-dimensional data sample (assuming an extremely large sample size $N$) such that the data are drawn from a uniform
  distribution on a circle (i.e., a ring; the boundary of a disc, but {\em not} its interior) with center as the origin and radius $r$.  Suppose you
  decide to fit a multivariate (2-dimensional) Gaussian distribution to this data by maximizing the likelihood function. The multivariate Gaussian is
  $P (x; \mu, C) := 1 / \sqrt {(2 \pi)^d |C|}) \exp (- 0.5 (x - \mu)^\top C^{-1} (x - \mu))$, where, for our case, dimension $d = 2$, $x$ and $\mu$
  are vectors of size $d \times 1$, $C$ is a matrix of size $d \times d$, and $|C|$ is the determinant of $C$.
  
  $\bullet$ Derive the mathematical formula, in terms of $r$, for the estimated mean and the estimated covariance matrix. You may use
  \url{http://www.ee.ic.ac.uk/hp/staff/dmb/matrix/calculus.html} to compute derivatives of the likelihood function with respect to the vector
  parameter $\mu$ and matrix parameter $C$.
  
  $\bullet$ Where is the mode of this Gaussian situated within $\mathbb{R}^2$~? Do you think this Gaussian fits the data well~? Is it a good model~?
  Why or why not~?
  
  $\bullet$ Generate a large sample that is uniformly distributed on a circle with center origin and radius $r$. Compute the maximum likelihood
  estimates for the mean and covariance and report them, along with the sample size. Do they match the theoretically predicted values~?
  
\item % Pareto prior
  
  (10 points) Suppose random variable $X$ has a uniform distribution over $(0, \theta)$, where the parameter $\theta$ is unknown. Consider a Pareto
  distribution prior on $\theta$, with a scale parameter $\theta_m > 0$ and a shape parameter $\alpha > 1$, as $P (\theta) \propto (\theta_m /
  \theta)^{\alpha}$ for $\theta \ge \theta_m$ and $P (\theta) = 0$ otherwise.
  
  $\bullet$ Find the maximum-likelihood estimate $\widehat \theta^{\text{ML}}$ and the maximum-a-posteriori estimate $\widehat \theta^{\text{MAP}}$.
  
  $\bullet$ Does $\widehat \theta^{\text{MAP}}$ tend to $\widehat \theta^{\text{ML}}$ as the sample size tends to infinity~? Is this desirable or
  not~?
  
  $\bullet$ Find an estimator of the mean of the posterior distribution $\widehat \theta^{\text{PosteriorMean}}$.
  
  $\bullet$ Does $\widehat \theta^{\text{PosteriorMean}}$ tend to $\widehat \theta^{\text{ML}}$ as the sample size tends to infinity~? Is this
  desirable or not~?
  
\end {enumerate}

\end {document}
